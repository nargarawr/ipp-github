 \section{Requirements}
 \subsection{Functional Requirements}
 \begin{enumerate}
 \item[1.] The user should be able to search by geographic region and discover submitted routes for that region.
 \item[2.] The user should be able to contribute routes.
 	\begin{enumerate}
 	\item[2.1.] Only the creating user should be able to modify these routes
 	\item[2.2.] The user should be able to decide on the visibility of this route (private or public)
 	\end{enumerate}
 \item[3.] The user should be able to interact socially with the route, including:
 	\begin{enumerate}
 	\item[3.1.] Comment on public routes
 	\item[3.2.] Recommend alternative routes
 	\item[3.3.] Share routes to external social media websites
 	\end{enumerate}
 \item[4.] Users should be able to create accounts and specify for example, name, age, email and location 
 \item[5.] The should be administrative users who have extra functionality available, including:
	 \begin{enumerate}
 		\item[5.1.] Manage users
 		\begin{enumerate}
 			\item[5.1.1.] Delete users
 			\item[5.1.2.] Update users
 			\item[5.1.3.] Create users
 		\end{enumerate}
 		\item[5.2.] Manage routes
 		\begin{enumerate}
 			\item[5.2.1.] Delete routes
 			\item[5.2.2.] Update routes
 		\end{enumerate}
 		\item[5.3.] Moderate comments
 		\begin{enumerate}
 			\item[5.3.1.] Delete comments
			\item[5.3.2.] Update comments
 		\end{enumerate}
 		\item[5.4.] Make announcements
 		\item[5.5.] Make backups of the website in a standard, compliant, free and open format
 		\item[5.6.] De-authorize active sessions
 		\item[5.7.] Lock the site and prevent access
 	\end{enumerate}
 \item[6.] Users should be able to export their routes to a FOSS
 \item[7.] Users should be able to fork other user's public routes (make a copy), and edit as they wish
 \item[8.] There should be a route editor component which allows the user's to create and specify their route
 \item[9.] The user should be able to log into their account and:
	 \begin{enumerate}
	 	\item[9.1.] Access and change personal information
	  	\item[9.2.] Access and edit their submitted routes
	 \end{enumerate}
 
 \end{enumerate}
 
 
 \subsection{Non-Functional Requirements}

\paragraph{Accessibility}\ \\
The proposed system is to be made available entirely on-line, allowing anyone with internet access, and a computer, to use the system. As the application will be web-based, the user is not required to download anything to start using it, this will make it more accessible for the user, and it will be easier for them to get started using the application. The only potential issue with a web-based application is if the server goes down, or if the domain expires. The server going down does not, unfortunately, have a solution, but keeping the domain should be relatively trivial.

\paragraph{Usability and Operability}\ \\
The project should be designed in such a way that users ranging from a low level of skill, to a high level of skill should be able to use it with little prior knowledge. There will be a dedicated help system present in the system, that will allow the user to get help, whenever they need it, without having to leave the system and check some external documentation. The project will be developed as a fully-responsive web application, meaning mobile devices will be fully supported.

\paragraph{Maintainability \& Documentation}\ \\
The system must be exceptionally well documented allowing for easy maintenance by an external developer. This includes both an easy to understand code structure, as well as commented code (specifically using JSDoc), which should be kept as modular as possible.  The system itself will have help available to the user, so that if they are confused, they can be guided in the right direction.

\paragraph{Quality}\ \\
As this system is to be used externally, and will be a representation of both myself, and the client, there are several quality issues that must be addressed. The system must be built so that it is robust, and works as the user expects, but it must also contain as few bugs as possible. Any bugs that are identified should be reportable to the maintainer, and be fixed as soon as possible.\ \\
\ \\
The quality of the code must also be considered. In this regard, the code will be written in as modular a way as possible, and useful comments will be provided to highlight the purpose of each function, and to illustrate any particularly complex code.

\paragraph{Resource Requirements and Constraints}\ \\
As this system is aimed at users with a mixture of skill levels, no assumptions can be made on the level of hardware that the users will possess. For this reason, the system will be designed to use as few resources as possible. Fortunately, due to being a web-based system, the load of the system would be fairly minimal anyway, as it is mostly loading only JavaScript and HTML 5. The only true computation takes place on the server, and thus would not be a concern for the user.\ \\
\ \\
The ability to load and save files potentially causes a problem, but only if the user has very little hard drive space, and attempts to save an extremely large file from the system. Unfortunately, there is nothing that can be done about this, although even very large systems will have a relatively small file size.\ \\
\ \\
The final concern is internet bandwidth which, whilst not a problem on desktops, will be a problem for mobile phones. This is why the amount of data sent to the user when they are navigating a route will be kept to a minimum, so that they do not use up their data allowance (or drain their battery).

\paragraph{Cross Platform Compatibility}\ \\
Due to the project being a web-based system (developed using HTML5, CSS3, and JavaScript), it is difficult for it not to be cross platform compatible. However, there are still a few issues that may have to be dealt with, especially when looking at different browsers that the user may be using. For this reason, research into how the system performs on different browsers will be important, so that it can be assured that any user using any browser will have full access to the system, as it was meant to be.

\paragraph{Security}\ \\
Security is a concern for this project, as the user's will be able to create accounts with the system, and therefore their data will need to be stored. This data will be stored in compliance with the Data Protection Act, and all passwords will be encrypted. Cookies may also be utilised on the site, so a cookie disclaimer will also be necessary. 

\paragraph{Disaster Recovery}\ \\
 The administrator should be able to take backups of the site in a standard, compliant, free and open format. They should also be able to de-authorize active sessions, and lock the website to prevent access. As far as the code base itself, the project will be stored both on the server, and in a Git repository, allowing for recovery if anything goes wrong.