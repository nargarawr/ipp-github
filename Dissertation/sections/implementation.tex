\section{Software Implementation}
In this section, the actual implementation of the software has been detailed, including: what tools were used in the implementation, how the software was implemented, and any issues that were encountered during the implementation process.

{\color{red}
	\begin{itemize}
		\item Screenshots of initial designs + justifications
		\item Screenshots of final designs + justifications + reasons for changes
		\item Screenshots of actual final system + reasons for changes
	\end{itemize}
}

\subsection{Key Implementation Decisions}
\label{sec:kid}
{\color{red}
	\begin{itemize}
		\item From the background research section, list all the technology I chose to use and why
	\end{itemize}
}

\subsection{Implementation Methodology}
{\color{red}
	\begin{itemize}
		\item To help manage the implementation of such a large piece of software, the adoption of some methodology was necessary. It was decided that the best methodology would be an agile one, with heavy use of Kanban, using the tips laid out by Henrik Kniberg \cite{kniberg2007scrum}. In order to accomplish this, at the beginning of the implementation stage, after the requirements specification had been detailed, the entire project was split into user stories. Each of these stories detailed a specific action that a user of the system would be able to accomplish, along with how long it should take to implement, how important it was, and a way of testing its completion. These stories were then organised onto a digital Kanban Board, using a service called Trello\footnote{\url{http://www.trello.com}}. \ \\
		\ \\
		Each week, a set of tasks would be selected to be worked on for that week. The amount of tasks selected would be dependent on how much was completed, on average, in the weeks before, so that reasonable estimates could be made (obviously excluding the first few weeks). This ensured a decent portion of work was being completed per week, and that progress was constant. During the week, tasks would be selected from the available pool, prioritising those that were prerequisites of others, or had a high importance, and would then be worked on until completion. After the completion of a task, a new task would be selected, and work would begin on this. This was an extremely effective method of managing the implementation, as any small tasks that were necessary could be added to the board, and there was an assurance they would eventually be completed, and nothing would be overlooked. It has also been shown that it is much easier to reach goals if they have been written down \cite{wilson2008goal}, which a Kanban Board was the perfect tool for.\ \\
		\ \\
		Also mention weekly meetings with max and use of Gantt chart. Also mention this is what I did at work and in my last diss and found it the best way to work for me?
	\end{itemize}

}
{\color{blue}
	full example of taking a requirement, making story, splitting into sub tasks (in table or screenshot format)
}

\subsection{Implementation of System Components}
{\color{red}
	\begin{itemize}
		\item Potentially don't need this
		\item Do last, look at last year's diss
	\end{itemize}
}


\subsection{Problems Encountered}
{\color{red}
	\begin{itemize}
		\item Look through problem log document and pick out key things, especially those with lessons
		\item What happened / what this affected / how the project was affected / what I would do differently / why it happened
		\item problem: lack of unit tests meant that the addition of features would break others. especially silly with the support Zend has for unit tests
	\end{itemize}
}