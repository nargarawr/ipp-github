\section{Motivation}
Niceway.to is an application envisioned by my client, Matthew Pike, who works for a small start-up in China. The project has already been worked on once before, but the end result was not to a standard that my client was content with. As a result of this, he would like for the project to be redesigned and recreated from scratch, as little of the original software is reusable.\ \\
\ \\
The main motivation of this project is to change how we think about driving: from simply a tool for travel, to an enjoyable recreational activity, where, instead of being focused solely on reaching our destination, we can take time to appreciate the beauty in the world around us. In a 1995 study, almost 90\% of motorists had experienced some form of road rage within the preceding 12 months, and 60\% admitted to losing their temper whilst behind the wheel\cite{joint1995road}. To facilitate the proposed shift in mentality, this project will be a tool that offers a collection of user submitted scenic driving routes, with a heavy focus on utilising the social characteristics of the Internet. The idea being that when a user wishes to travel somewhere, they could do so by travelling a route lesser known to them. This route may be slower, but would make up for the time lost, by providing a visually enriching experience that the driver would have otherwise been deprived of, and to help avoid common annoyances on the road. This would also help to relieve the monotony of driving the same, mundane, routes repeatedly - which has been shown to have serious implications in terms of accident causation\cite{thiffault2003monotony}. \ \\
\ \\
A big part of this project, which my client has stressed, is to foster a community of travel enthusiasts. Specifically the ability to have a social commentary surrounding each route, where registered users of the application are able to express their opinions on the content, give a numeric rating for the content, as well as share the content (both internally and externally). The reason for this is because a user is far more likely to return to, and remain engaged with, a website if there are other users doing the same, especially if they are directly communicating with those other users {\color{red}REF}. Allowing users to submit their own content coaxes them into feeling more of a connection to the site, and will increase their chance of returning (so that they can check how well received their content is). It has been shown that feedback is a useful tool to boost engagement\cite{o2008user} and hopefully this will encourage users to associate the site with positive experiences, and inspire them to produce quality content (in order to receive more of this feedback).\ \\
\ \\
In order for this community to thrive, it is vital that the system is simple to use, open to users of all skill groups, and easy to access. Therefore, it is important that HCI principles are kept in mind throughout every step of the design and implementation processes. Key to a good design, is simplicity. This is because complex user interfaces frustrate users, and can deter them from returning. Studies have shown that users lose more than 40\% of their time to frustration, and in most cases the user ends up angry at themselves, angry at the computer, or left with a feeling of helplessness\cite{lazar2006workplace}.\ \\
\ \\
In addition to all of these reasons, I feel personally motivated to ensure that this project succeeds. As someone who does not currently drive, I find myself in need of others to drive me when public transport proves inadequate. As a passenger, I will often observe the driver becoming evermore agitated with other drivers on the road, and seeing this problem first hand helps enforce my feelings of the importance of solving it. Driving is a freedom, but one that is being squandered and perceived more as a chore than an enjoyable activity.