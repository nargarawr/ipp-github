\section{Testing of the Project}

\subsection{Functional Testing}
{\color{red}
	\begin{itemize}
		\item In this section, each of the functional requirements laid out in section \ref{sec:funcs} have been evaluated in turn, to ensure the system meets them. Knowledge of the inner workings of the system is not actually necessary to understand these tests, as they simply check whether functionality is present, and are not concerned as to how the system actually implements it (this is known as black box testing \cite{beizer1995black}). A complete listing of all the tests conducted, and their results, can be found in appendix \ref{app-ctl}.\ \\
		\ \\
		some interesting thing to point out... ?
	\end{itemize}
}

\subsection{Non-Functional Testing}
{\color{red}
	\begin{itemize}
		\item Look at non-functional requirements and talk about if they were met
		\item Just like last year's diss
	\end{itemize}
}
\paragraph{Accessibility}\ \\
{\color{red} What this met? If so, \textbf{how}?}

\paragraph{Usability and Operability}\ \\
{\color{red} What this met? If so, \textbf{how}?}

\paragraph{Maintainability \& Documentation}\ \\
{\color{red} What this met? If so, \textbf{how}?}

\paragraph{Quality}\ \\
{\color{red} What this met? If so, \textbf{how}?}

\paragraph{Resource Requirements and Constraints}\ \\
{\color{red} What this met? If so, \textbf{how}?}

\paragraph{Cross Platform Compatibility}\ \\
{\color{red} What this met? If so, \textbf{how}?}

\paragraph{Security}\ \\
{\color{red} What this met? If so, \textbf{how}?}

\paragraph{Disaster Recovery}\ \\
{\color{red} What this met? If so, \textbf{how}?}

\subsection{User Feedback Testing}
{\color{red} ask max ethics for this type of testing}\ \\
\ \\
{\color{cyan} 
usability testing involved getting actual users to use to system. THere were 5 tests, that represented the five main actions that a user would perfgorm on the application. It was vital that these were easy to follow and error free, to not hinder the user. \ \\
\ \\
Make sure to include that the first thing user 1 found was the return on route search\ \\
\ \\
Write up what I learned from each task/user, and any usability/bugs identified by the user, and the time it took them to complete each task\ \\
\ \\
Bar chart of min/max/average time taken to complete each task against one another. (on seperate plots) - the thing to look for is the range. If this is high (but not because of outliers), it is hard. If the range is low, the task is easier, because even those with low skill levels understand how to do it. Results can be included in full in the appendix in the format:\ \\
\ \\
\begin{tabular}{lll}
\textbf{User 1}  & & \\
\textbf{Task}  & Time & comments/bugs \\
1. Route Search & 45 seconds& BUG: can't press enter to search\\
\end{tabular}
\ \\
\ \\
the aim of these tests is to find \textbf{usability problems} - rite method. 
For these tests I selected a sample of 15 (low to high skill leve) users of a range of ages from 22-50 to complete a series of simple tasks and recorded the times they took, and the comments they made. The tests were A, B, C, D and E, with a full list of the tasks to complete listed in appendix F. These five tasks were the five core tasks that users of niceway.to would be performing and therefore it was vital that they were easy to understand and easy to complete (which is why the instructions are as vague as possible). The RITE method was implemented, which is rapid iterative testing and evaluation, and means that you x, y and z. In this case, after each test, the comments of the users were considered, and any bugs were fixed before the start of the next test. This meant that the next participants would be identifying some other issue, or potentially be identifying an issue with a previous change. Some of these tasks were purposefull extremely simple, because x y and z.
}

{\color{blue}
An important part of evaluating the usability and accessibility of any software system is to have real world users attempt to actually use it - The usability engineering life cycle \cite{nielsen1992usability} - expand this
}

{\color{red}
	how do we actually write this up? - mention the methodoloy (already done), and then go through one specific case / or task? maybe a feature that was added
}