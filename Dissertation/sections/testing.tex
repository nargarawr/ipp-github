\section{Testing of the Project}

\subsection{Functional Testing}
{\color{red}
	\begin{itemize}
		\item In this section, each of the functional requirements laid out in section \ref{sec:funcs} have been evaluated in turn, to ensure the system meets them. Knowledge of the inner workings of the system is not actually necessary to understand these tests, as they simply check whether functionality is present, and are not concerned as to how the system actually implements it (this is known as black box testing \cite{beizer1995black}). A complete listing of all the tests conducted, and their results, can be found in appendix \ref{app-ctl}.\ \\
		\ \\
		some interesting thing to point out... ?
	\end{itemize}
}

\subsection{Non-Functional Testing}
{\color{red}
	\begin{itemize}
		\item Look at non-functional requirements and talk about if they were met
	\end{itemize}
}

\subsection{User Feedback Testing}
An important part of evaluating the usability and accessibility of any software system is
to have real world users attempt to actually use it - The usability engineering life cycle \cite{nielsen1992usability}
{\color{red}
	\begin{itemize}
		\item user feedback / questionnaire / focus group / test users? + their feedback
		\item the aim of these tests is to find \textbf{usability problems}
		\item RITE method for discovering bugs
		\item proof of people that are shit finding it easy to use
		\item average time for users to complete short tasks?
	\end{itemize}
}