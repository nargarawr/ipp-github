\section{Introduction}
Fuzzy logic is an ever expanding field, and as such, the tools we are using to work in this field should also be expanding. It is also important that the merits of fuzzy logic are made apparent to those other than the experts of this field, as this would help to produce more advanced control systems in the future.\ \\
\ \\
Many software systems for working with fuzzy logic have already been produced, of which many different approaches have been attempted, and been successful to various degrees. Examples of such systems include: The MATLAB Fuzzy Toolbox\footnote{\url{http://www.mathworks.co.uk/products/fuzzy-logic/}}, An R Package named  FuzzyToolkitUoN\footnote{\url{http://cran.r-project.org/web/packages/FuzzyToolkitUoN/index.html}}, XFuzzy\footnote{\url{http://www2.imse-cnm.csic.es/Xfuzzy/}}, and fuzzyTECH\footnote{\url{http://www.fuzzytech.com/}} (a more comphrehensive overview of these systems can be found in section \ref{sec:existing-systems}).\ \\
\ \\
These system are all worthwhile pieces of software, and they fulfil their main objective of allowing for the creation of fuzzy systems. However, whilst researching these systems as part of my second year group project at the University of Nottingham (and actually working on one, in the case of FuzzyToolkitUoN), I noticed that there were two key flaws that the majority of popular fuzzy software systems suffered from: difficulty of use, or difficulty of access (or even both). \ \\
\ \\
The main objective of this project is to produce a software solution for the creation, manipulation, and inferencing of a fuzzy logic system, which is accessible online. With a specific focus on solving the issues that are faced by fuzzy logic software systems that are currently used (difficulty of access and use). \ \\
\ \\
Many different techniques will be employed in solving these fundamental problems, to hopefully create a system that is as easy to use, and as easy to access, as possible. Some of these techniques will include: online access; the ability to work with multiple file types, for cross compatibility; an intuitive design; unrestricted navigation, giving the user complete control and freedom; a dedicated, unobtrusive help system, to offer help to those that need it, but not to bother those that do not; and to build it in a way that allows for future expansions. A comprehensive list of all aspects of the software system can be found in section \ref{sec:spec}.\ \\
\ \\
It could be argued that \textit{another} fuzzy logic software system is not necessary, as it has been demonstrated that there are already many systems available. However, the currently available software suffers from the key issues identified above, and this project aims to resolve these issues, and attempt to spread the influence of fuzzy logic to those other than experts in the field.\ \\
\ \\
There will, however, be certain areas that this software system will \textit{not} be focusing on, as these are not relevant to the question posed in this research. Namely, this project will not be focusing on higher levels of fuzzy logic; it will only be focused on type-1. This is because the leap in difficulty from type-1 to type-2 fuzzy logic is very large, and type-2 is simply not a concept that is suitable or appropriate to introduce beginners to. More on this topic, including a definition of both terms, can be found in section \ref{sec:type2}.