\section{Introduction}
\label{sec:intro}
In recent years, the technologies behind satellite navigation and routing services have greatly advanced, allowing them to produce the quickest route between two points in only a few seconds, even with technological limits \cite{lou2009map}. As a result, travelling via car is quicker and easier than ever before, and many drivers are now focused solely on reaching their destination as quickly as possible. This has resulted in a shift in the mindset of society where the scenery that we pass on the road is simply a buffer between segments of our day, and its beauty is left unappreciated. This mentality promotes a culture of instant gratification, impatience, and self-involvement among the driving community, which has a huge detrimental effect on drivers, where any interruptions on their journey are cause for anger. In fact, it has been shown that since 1990, incidents of aggression during driving has risen 51\% \cite{vest1997road}. \ \\
\ \\
Some research has already been completed in an attempt to shift the focus of driving from simply travel, to also being an enjoyable recreational activity. This includes work such as recommending ``nice'' routes (determining this using social network data \cite{peregrino2012mapping}\cite{van2011time}\cite{quercia2014shortest}) and how to personalise routes that have been deemed ``efficient'' \cite{chen2011discovering}. Outside of the world of research, many services already exist that provide users with a collection of scenic routes between two locations. The purpose of these services is to encourage drivers to enjoy the experience of driving more, and be exposed to more of their surroundings. Examples of these include Google's ``My Maps''\cite{url2015gmaps}, MADMAPS\cite{url2015madmaps}, and MyScenicDrives\cite{url2015myscenicdrives}, which are discussed further in section \ref{sec:existing-systems}.\ \\
\ \\
Unfortunately, these systems have some flaws that mean they do not fully solve the above problem, and therefore have not made a huge impact. The largest two being the method of delivery, and the inability for of users to contributions content. Google's ``My Maps'', and MyScenicDrives are optimised for desktop browsing and their support for mobile devices is limted. This is a large portion of users lost, considering that in the United Kingdom 33\% of Internet users believe that smartphones are the most important device for going online\cite{ofcom2015comms}. MADMAPs, whilst mostly focusing on the selling of physical maps, does also provide a mobile application, but this is clunky, and poorly designed. Alongside these larger flaws, some other minor flaws are also present, including small user bases, primitive search functionality, slow and unresponsive webpages, and some services costing money.\ \\
\ \\
Niceway.to has three main aims: to build a community of travel enthusiasts, improve the travelling experience of those that use it, and allow for users of all skill groups to access it. It will provide a way for users to discover scenic and visually interesting routes between two locations, all of which have been provided by other members of the community. These routes will each contain a social commentary, with users being able to rate them, comment on them, and share them (these will be incentives for users to provide quality content, and to remain loyal to the site). To address the problem of previous software systems, mainly the method of delivery, it will be built as a fully-responsive web-application that functions equally well on desktop and mobile devices. \ \\
\ \\
As a final note, it should be mentioned that this project will not be focussing on the classification of whether or not routes are ``nice'' or ``scenic''. Instead, the content will be entirely user driven, with the assumption that they would only contribute routes that are interesting and visually appealing. To further encourage this, a rating system for routes will be implement, so that ``better'' routes are more visible than those deemed less so.
