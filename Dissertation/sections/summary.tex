\section{Summary \& Personal Evaluation}
Despite some ups and downs through the lifetime of the project, I would categorically claim it was a success. This comes from the meeting of all of the functional and non-functional requirements, and ample user feedback claiming that the website is aesthetically pleasing and simple to use (regardless of user skill level).
\ \\
\ \\
Replacing the title held by my dissertation, this software was the largest single piece of software I had ever constructed by myself, with a code base of over 12,000 lines, split over almost 80 files. I used several methods of planning and time management (many of which were things that I learnt worked well from my dissertation), which meant that I was often ahead of my initial project time plan (which was well received by my client, and meant I had more time to revise the project from feedback received). This included the Gantt Chart, for the ``big picture'' planning of large sections of work, and Trello for the day-to-day,  granular tasks. I feel that one of the main reasons I worked so efficiently on this project is that it was fun to work on, as I enjoy both the languages that were used (PHP and JavaScript), and the platform utilised (responsive web applications).\ \\
\ \\
As elaborated in section \ref{sec:problems}, one of the key flaws in the project was the lack of unit tests being utilised. This caused huge issues in the project when new features were being added, with many unseen bugs being introduced. Time was then required later in the project (when these bugs were eventually discovered) to go back and fix these issues. The reason for not using unit tests was due to my eagerness to get started on the project, causing me to write sloppy code during the prototyping stage, that did not have unit tests. Unfortunately, this prototype code was carried forward into the actual implementation of the project, and therefore also this cavalier attitude towards unit tests. This meant that the project could progress at an accelerated rated, but the drawback was the introduction of these bugs that would cause issues later.\ \\
\ \\
If given the opportunity to restart this project from scratch, there are several changes I would implement to the development process. The first, and potentially most obvious, would be the use of unit tests for all aspects of the project, as this would have made integration and regression tests much simpler, and allowed me to identify new bugs quicker. In addition to this, I would have spent more time, and gone more in depth, during the research portion of the project. In this section, several pieces of software, as well as different tools I could use were evaluated. However, only three only pieces of software and only front end and back end tools were looked at. I feel as though the project would have benefited from looking at a greater range of other software systems (to determine what is good about them, and what things to avoid), and well as looking at more different kinds of tools to use (for instance: which mapping service to use, which routing service to use, and what JavaScript libraries would be useful). This would have meant that these choices didn't need to be rushed later in the project, and I could have picked the right tool for the job, in each of these cases.