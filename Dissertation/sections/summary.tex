\section{Summary \& Personal Evaluation}
Personally, I feel as thought the project was a success, as it solved the two main issues it was conceived to solve, it met all the functional and non-functional requirements set out, and user feedback was almost entirely positive. Another reason that I believe the project was successful is that I have used the software myself, and found it a pleasurable experience. During the life time of the project, I enrolled in a module at the University of Nottingham known as \emph{Fuzzy Sets and Fuzzy Systems}\cite{uon2014fuz}. In this module, I was required to use FuzzyToolkitUoN to construct a system that would advise a doctor on how urgently a patient should be sent to the hospital, based on their heart rate and temperature. However, I found FuzzyToolkitUoN cumbersome to use (which, considering I helped to produce it, shows how poor of a piece of software it is), and instead used the prototype version of my dissertation software. This greatly sped up the process of completing this work, and gave me a unique experience to test my system as an end user.\ \\
\ \\
This was the largest project that I had ever worked on by myself, but I felt as though I rose to the challenge, and employed several clever methods in order to help manage my time, and the work to be completed (such as a Kanban Board, and the use of a Gantt Chart). I did not always stick to the time scale set out by my Gantt chart, due to other work, and commitments, but the combination of the Gantt chart to track the ``big-picture'', and the Kanban board to track individual features made the management of the project much simpler.\ \\
\ \\
One of the areas I feel as though was weaker within the project, was the research segment, or more specifically, research into software that could be used. Whilst multiple examples were evaluated against one another, this ``research'' was more of a superficial overview of the software, instead of an in-depth analysis. As a result of this, half way through the implementation, many issues began to arise with the tools chosen. The most prominent of these was the inability to properly construct dynamic user interfaces and have these interface with R-Shiny, and the numerous bugs and issues discovered whilst using FuzzyToolkitUoN. I also feel as though the testing process could have been much more rigorous, as numerous issues arose throughout the life time of the software development, and solving them was often time consuming, and cumbersome.\ \\
\ \\
If I could work on this project again, from the beginning, there are several changes that I would make. The first of these would be a more rigorous evaluation of the tools that would be used, so that the major issues that arose with this project would not have done so. The other major change would be to implement Unit Testing and Test Driven Development from the start of the project, so that the code quality could be ensured throughout, and bugs easily identified. This includes integration, and regression testing, which would have made the addition of new features, and any extensions to the software, much simpler and easier to debug.