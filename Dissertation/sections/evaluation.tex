\section{Evaluation of the Project}

\subsection{Functional Testing}
{\color{red}
	\begin{itemize}
		\item In this section, each of the functional requirements laid out in section \ref{sec:funcs} have been evaluated in turn, to ensure the system meets them. Knowledge of the inner workings of the system is not actually necessary to understand these tests, as they simply check whether functionality is present, and are not concerned as to how the system actually implements it (this is known as black box testing \cite{beizer1995black}). A complete listing of all the tests conducted, and their results, can be found in appendix \ref{app-ctl}.
	\end{itemize}
}

\subsection{Non-Functional Testing}
{\color{red}
	\begin{itemize}
		\item Look at non-functional requirements and talk about if they were met
	\end{itemize}
}

\subsection{User Feedback Testing}
{\color{red}
	\begin{itemize}
		\item user feedback / questionnaire / focus group / test users? + their feedback
	\end{itemize}
}

\subsection{Successes and Limitations of the Project}
{\color{red}
	\begin{itemize}
		\item As a result of the test...
		\item x was good
		\item y was bad
	\end{itemize}
}
